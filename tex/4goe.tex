
\section*{\centering Summary}

The Great Oxidation Event (GOE), arguably the most important event to occur on Earth since the origin of life, marks the time when an oxygen-rich atmosphere first appeared. However, it is not known whether the change was abrupt and permanent or fitful and drawn out over tens or hundreds of millions of years. Here, we developed a 1-D time-dependent photochemical model to resolve time-dependent behavior of the chemically unstable transitional atmosphere as it responded to changes in biogenic forcing. When forced with step-wise changes in biogenic fluxes, transitions between anoxic and oxic atmospheres take between only $10^2$ and $10^5$ years. Results also suggest that O$_2$ between $\sim10^{-8}$ and $\sim10^{-4}$ mixing ratio is unstable to plausible atmospheric perturbations. For example, when atmospheres with these O$_2$ concentrations experience fractional variations in the surface CH$_4$ flux comparable to those caused by modern Milankovich cycling, oxygen fluctuates between anoxic ($\sim10^{-8}$) and oxic ($\sim10^{-4}$) mixing ratios. Overall, our simulations are consistent with possible geologic evidence of unstable atmospheric O$_2$, after initial oxygenation, which could occasionally collapse from changes in biospheric or volcanic fluxes. Additionally, modeling favors mid-Proterozoic O$_2$ exceeding $10^{-4}$ - $10^{-3}$ mixing ratio, otherwise O$_2$ would periodically fall below $10^{-7}$ mixing ratio, which would be inconsistent with post-GOE absence of sulfur isotope mass-independent fractionation.

\section{Introduction}

Abundant atmospheric O$_2$ at 21\% by volume is the most distinctive and consequential feature of Earth's atmosphere. Produced by cyanobacteria, algae and plants, O$_2$ is a clear sign of our biosphere that is detectable across interstellar space by telescopic spectroscopy \cite{Meadows_2018}. Oxygen permits aerobic respiration, the only known metabolism  with sufficient energy yield that can sustain complex animal life \cite{Catling_2005}. However, for about the first half of Earth's 4.5-billion-year-old history, the atmosphere had negligible O$_2$ \citep[e.g.][]{Catling_2020}. This changed $\sim$2.4 billion years ago.

The timing of the GOE, and magnitude of atmospheric O$_2$ concentrations before and after the GOE can be constrained by the geologic record of sulfur isotopes in combination with photochemical models. Archean and earliest Proterozoic sedimentary minerals contain sulfur isotopes with characteristic mass independent fractionation (MIF) which abruptly disappears 2.4 billion years ago \citep{Warke_2020}. Sulfur MIF in marine sediments likely requires that atmospheric photochemistry produce elemental sulfur, S$_8$ \citep[for explanation, see the introduction in][]{Zahnle_2006} \citep{Pavlov_2002,Farquhar_2000}. \citet{Zahnle_2006} used a 1-D photochemical model to show that atmospheric S$_8$ production only occurs when atmospheric O$_2$ is below $\sim 2 \times 10^{-7}$ mixing ratio. An often cited threshold of $2 \times 10^{-6}$ was from an earlier photochemical model that did not simulate atmospheres with surface O$_2$ mixing ratios between $2 \times 10^{-6}$ and $\sim 10^{-15}$ \citep{Pavlov_2002}. Therefore, the disappearance of the sulfur isotope MIF signal at 2.4 Ga is strong evidence that O$_2$ first rose above $2 \times 10^{-7}$ mixing ratio.

Geologic evidence may suggest that the GOE was not a single monotonic rise of oxygen but characterized by oscillations. Using U-Pb dating, \citet{Gumsley_2017} updated the chronology of sulfur isotope MIF in the stratigraphic record, finding evidence for two oxic-to-anoxic transitions between $\sim 2.4$ and $\sim 2.3$ Ga. More recently, \citet{Poulton_2021} report 2.3 - 2.2 Ga marine sediments with sulfur isotopes consistent with approximately five oxic-to-anoxic transitions. Fluctuating O$_2$ levels coincide with three to four widespread glaciations, indicating extreme climate instability \citep{Rasmussen_2013}. Overall, geochemical evidence tentatively suggests that O$_2$ concentrations and climate were unstable for 200 million years until $\sim 2.2$ Ga, which marks the most recent estimated timing of the permanent oxygenation of the atmosphere \citep{Poulton_2021}. However, interpretations of oscillating O$_2$ have been questioned \citep{Izon_2022}. While the geologic evidence for the O$_2$ oscillations remains equivocal, the data has raised significant questions regarding the feasibility and timescales for Earth's great oxidation. Some have argued that the oxygen-rich atmosphere is more stable than an oxygen-poor atmosphere \citep{Goldblatt_2006}, which favors a single rise of O$_2$ instead of O$_2$ oscillations.

Evidence for O$_2$ instability and the time-dependent behavior of O$_2$ concentrations have not been reconciled with atmospheric photochemical models. All previous models treated the GOE as successive photochemical steady states \citep{Kasting_1980,Segura_2003,Pavlov_2001,Pavlov_2002,Zahnle_2006,Bethan_2021,Claire_2014,Izon_2017,Kurzweil_2013}. A photochemical steady state occurs when no atmospheric species changes concentration over time because their production and loss from reactions and surface sources (e.g. volcanoes or biology) are balanced. Such steady state calculations have been crucial for understanding the GOE by contextualizing sulfur isotope MIF observations \citep{Pavlov_2002,Zahnle_2006}, and establishing the relationship between atmospheric O$_2$ concentrations and the degree to which O$_3$ blocks UV photons from Earth's surface (i.e. O$_3$ shielding) \citep{Kasting_1980,Pavlov_2001,Bethan_2021}, but they do not evaluate time-dependent changes and transient imbalances, or characteristic timescales.

Several theories for the rise of O$_2$ suggest that it relied on a global redox titration over $10^8$ - $10^9$ years involving oxidation of the upper mantle and/or crust, plausibly driven by hydrogen escape, which led to a tipping point where the source flux of O$_2$ exceeded a kinetically rapid O$_2$ sink from volcanic and metamorphic reductants \citep{Catling_2001,Claire_2006,Kadoya_2020,Holland_2002,Kasting_1993}. Beyond the tipping point, O$_2$ flooded the atmosphere, reaching a new, long-term balance limited by oxidative weathering. 

Here, we developed a novel, time-dependent 1-D photochemical model capable of investigating changes of O$_2$ at the tipping point itself over timescales $10^2$ - $10^5$ years rather than the longer-term planetary changes which initiated the GOE. We simulate changing O$_2$ as a time-dependent evolution, in contrast to the steady-state approach used in previous studies \citep[e.g.][]{Kasting_1980}, because O$_2$ can change on relatively rapid timescales that are not well characterized by steady-states. With our model, we compute the time required for an anoxic-to-oxic atmospheric transition, and the time required for de-oxygenation. Additionally, we investigate the stability of O$_2$ concentrations against perturbations to surface gas fluxes produced by biology. Finally, we use our model results to better constrain O$_2$ levels and stability during the GOE (starting $\sim 2.4$ Ga), and during the mid-Proterozoic eon (1.8 to 0.8 Ga).

\section{Results}