
\section*{\centering Summary}

The Great Oxidation Event (GOE), arguably the most important event to occur on Earth since the origin of life, marks the time when an oxygen-rich atmosphere first appeared. However, it is not known whether the change was abrupt and permanent or fitful and drawn out over tens or hundreds of millions of years. Here, we developed a 1-D time-dependent photochemical model to resolve time-dependent behavior of the chemically unstable transitional atmosphere as it responded to changes in biogenic forcing. When forced with step-wise changes in biogenic fluxes, transitions between anoxic and oxic atmospheres take between only $10^2$ and $10^5$ years. Results also suggest that O$_2$ between $\sim10^{-8}$ and $\sim10^{-4}$ mixing ratio is unstable to plausible atmospheric perturbations. For example, when atmospheres with these O$_2$ concentrations experience fractional variations in the surface CH$_4$ flux comparable to those caused by modern Milankovich cycling, oxygen fluctuates between anoxic ($\sim10^{-8}$) and oxic ($\sim10^{-4}$) mixing ratios. Overall, our simulations are consistent with possible geologic evidence of unstable atmospheric O$_2$, after initial oxygenation, which could occasionally collapse from changes in biospheric or volcanic fluxes. Additionally, modeling favors mid-Proterozoic O$_2$ exceeding $10^{-4}$ - $10^{-3}$ mixing ratio, otherwise O$_2$ would periodically fall below $10^{-7}$ mixing ratio, which would be inconsistent with post-GOE absence of sulfur isotope mass-independent fractionation.

\section{Introduction}

Abundant atmospheric O$_2$ at 21\% by volume is the most distinctive and consequential feature of Earth's atmosphere. Produced by cyanobacteria, algae and plants, O$_2$ is a clear sign of our biosphere that is detectable across interstellar space by telescopic spectroscopy \cite{Meadows_2018}. Oxygen permits aerobic respiration, the only known metabolism  with sufficient energy yield that can sustain complex animal life \cite{Catling_2005}. However, for about the first half of Earth's 4.5-billion-year-old history, the atmosphere had negligible O$_2$ \citep[e.g.][]{Catling_2020}. This changed $\sim$2.4 billion years ago.

The timing of the GOE, and magnitude of atmospheric O$_2$ concentrations before and after the GOE can be constrained by the geologic record of sulfur isotopes in combination with photochemical models. Archean and earliest Proterozoic sedimentary minerals contain sulfur isotopes with characteristic mass independent fractionation (MIF) which abruptly disappears 2.4 billion years ago \citep{Warke_2020}. Sulfur MIF in marine sediments likely requires that atmospheric photochemistry produce elemental sulfur, S$_8$ \citep[for explanation, see the introduction in][]{Zahnle_2006} \citep{Pavlov_2002,Farquhar_2000}. \citet{Zahnle_2006} used a 1-D photochemical model to show that atmospheric S$_8$ production only occurs when atmospheric O$_2$ is below $\sim 2 \times 10^{-7}$ mixing ratio. An often cited threshold of $2 \times 10^{-6}$ was from an earlier photochemical model that did not simulate atmospheres with surface O$_2$ mixing ratios between $2 \times 10^{-6}$ and $\sim 10^{-15}$ \citep{Pavlov_2002}. Therefore, the disappearance of the sulfur isotope MIF signal at 2.4 Ga is strong evidence that O$_2$ first rose above $2 \times 10^{-7}$ mixing ratio.