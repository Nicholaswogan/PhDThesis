Chapters \ref{ch:2} through \ref{ch:4} study how life impacted the early Earth's atmosphere to inform the search for life on exoplanets. Chapter \ref{ch:2} computes the change in Earth's atmosphere-ocean chemical disequilibrium when life emerged. I find that the prebiotic Earth had a relatively big disequilibrium due to the coexistence of water vapor, CO, H$_2$ and CO$_2$. This atmospheric free energy was produced by volcanism. When life emerged, it consumed much of the free energy in the prebiotic atmosphere, replacing it with a smaller disequilibrium between biogenic waste gases: CO$_2$, CH$_4$, N$_2$ and liquid water. My finding that life destroyed much of the atmospheric free energy on prebiotic Earth contrasts the traditional view, proposed by \citet{Lovelock_1965}, that big disequilibrium should be generally associated with life. The early Earth suggests a more subtle relationship between life and atmospheric free energy.

I suggest that the disequilibrium-life relationship can be understood by considering the ``edibility'' of the disequilibrium in terms of reaction activation energy. Life consumed the prebiotic disequilibrium between H$_2$ and CO$_2$ or CO and H$_2$O because the reactions combining these species had relatively small activation energy barriers that could be overcome by enzymes. In contrast, the free energy in Earth's atmosphere and ocean since the origin of life had big activation energy barriers that were insurmountable by biological catalysis. In other words, the prebiotic disequilibrium was ``edible'' and the disequilibrium present since biopoiesis was not. On this basis, I argue that big ``edible'' disequilibrium (e.g., the coexistence of volcanic H$_2$ and CO$_2$ or CO and H$_2$O) should be considered an anti-biosignature in exoplanet atmospheres. Life on an inhabited planet would consume this free lunch.

Life's presence or absence on an exoplanet cannot be definitively deduced by mere detections of ``edible'' or ``inedible'' chemical disequilibrium. We must also consider the surface fluxes of biogenic gases, and whether they might be mimicked by abiotic processes. This reality motivated Chapter \ref{ch:3}, which considers the surface fluxes of methane required to sustain the CH$_4$-CO$_2$ disequilibrium biosignature characteristic of the Archean Earth, and whether these fluxes might be imitated by magmatic volcanic outgassing on an exoplanet. Over a wide parameter space, my model of volcanic outgassing suggests that big methane fluxes comparable to biological fluxes are unlikely. In the rare circumstances where volcanic methane appears possible in our model, volcanoes also produce large amounts of carbon monoxide. Therefore, a genuine CH$_4$-CO$_2$ biosignature should most likely coincide with the lack of atmospheric CO.

Chapter \ref{ch:4} investigates atmospheric O$_2$, Earth's most renowned sign of life, during the Great Oxidation Event around 2.4 billion years ago. I use a novel time-dependent photochemical model to show that O$_2$ is unstable for concentrations between $10^{-8}$ to $10^{-4}$ mixing ratio. Natural perturbations to the atmosphere (e.g. volcanic eruptions, or changes in climate) drive this instability because they can cause O$_2$ between $10^{-8}$ and $10^{-4}$ mixing ratio to change by $\sim 4$ order of magnitude. My modeling results provide a physical explanation for geologic evidence of O$_2$ oscillations preserved in the geologic record \citep{Poulton_2021}. Furthermore, the possibility of unstable O$_2$ requires that oxygen was bigger than $10^{-4}$ mixing ratio in the mid-Proterozoic eon. If O$_2$ concentrations were lower (e.g. $10^{-5}$ mixing ratio), then the natural perturbations to Earth would cause O$_2$ to occasionally drop below $10^{-7}$ mixing ratio, which would be incompatible with the geologic record of sulfur isotopes. My new constraint on Proterozoic O$_2$ ($> 10^{-4}$ mixing ratio) has implications for the detectability of the Proterozoic biosphere if it were to manifest on an exoplanet.

The likelihood of finding life on exoplanets is inextricably linked to the likelihood of the emergence of life. This fact motivates Part \ref{pt:2} of this thesis, which attempts to understand the atmospheric composition and climate on the Hadean Earth when life first began. I focus on the RNA-world hypothesis for the origin of life, which require nitriles (e.g. HCN and HCCCN) to abiotically synthesize the first molecules capable of encoding genetic information. In Chapter \ref{ch:5}, I use atmospheric chemistry and climate models to show that massive asteroid impacts could create H$_2$- and CH$_4$-rich atmospheres that would photochemical produce HCN and HCCCN. Nitrile production is most efficient for impactor masses larger than $5 \times 10^{20}$ to $4 \times 10^{21}$ kg (570 to 1330 km diameter). Smaller impacts generate about four orders of magnitude less HCN and fail to produce HCCCN. The minimum impact mass to cause big prebiotic molecule production ($> 5 \times 10^{20}$ vs. $> 4 \times 10^{21}$ kg) depends on a variety of assumption, including whether nickel-catalyzed methane production occurs in the post-impact environment, and how effectively an impactor can reduce the Hadean atmosphere.

My models suggest that photochemically generated nitriles would dissolve in rain droplets or otherwise be incorporated in aerosols that would fall into waterbodies on land. 
At this point, origin of life chemistry would have some of the required ingredients, but perhaps an unfavorable climate. With the caveat of uncertain planetary albedo from clouds, the reducing atmospheres that make nitriles would have $> 360$ K surface temperatures because of a strong H$_2$-H$_2$ collision-induced absorption greenhouse effect. Such warm temperatures could be a problem for the longevity of RNA and its precursors. As a solution, I suggest that prebiotic molecules are stockpiled and preserved in salts until the H$_2$-rich atmosphere escapes to space. After H$_2$ escapes, the climate would be temperate. In this cooler climate, stockpiled nitriles could then be released into waterbodies on land, providing an ideal setting for prebiotic chemistry.

The scenario in the previous paragraph is a hypothesis with only some support in the literature. This idea should be tested by further modeling and experiments, including work that determines the efficiency of preserving cyanides in minerals and salts in warm post-impact conditions.

The final chapter of this thesis considers the following thought experiment: Suppose life began in the wake of a post-impact reducing atmosphere. Given our understanding for what is needed to produce a post-impact highly reducing atmosphere from Chapter \ref{ch:5} and our understanding of Earth's impact history, when was the most likely timing of life's emergence on the early Earth? Furthermore, do all possible stochastic impact realizations result in biopoiesis, or do a significant fraction fail to make origin of life conditions favorable? 

Chapter \ref{ch:6} attempts to answer these questions using Monte-Carlo simulations of Earth's impact history. Based on Chapter \ref{ch:5}, I assume the minimum impact mass to generate significant prebiotic nitriles and an origin of life is $> 5 \times 10^{20}$ kg in an optimistic case and $> 4 \times 10^{21}$ kg in a pessimistic scenario. For either assumption, I find that the last life-starting impact is generally favored early in the Hadean eon ($\sim 4.35$ Ga), however the 95\% uncertainty is big and spans about 4.45 to 3.9 Ga. Only a fraction of my simulated impact histories result in an origin of life. In some cases, an impact of sufficient mass to make prebiotic molecules does not occur, or alternatively does occur, but is followed by an impact that vaporizes the ocean, which I assume sterilizes the planet. In an optimistic case, life begins in $\sim 90\%$ of stochastic impact realizations. In my most pessimistic case, life's emergence still has a 20\% chance of success. If this thought experiment has merit, then these results bode well for life on rocky exoplanets given that they all formed from accretionary impacts.
