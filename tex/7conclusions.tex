Chapters \ref{ch:2} through \ref{ch:4} study how life impacted the early Earth's atmosphere to inform the search for life on exoplanets. Chapter \ref{ch:2} computes the change in Earth's atmosphere-ocean chemical disequilibrium when life emerged. I find that the prebiotic Earth had a relatively big disequilibrium due to the coexistence of water vapor, CO, H$_2$ and CO$_2$. This atmospheric free energy was produced by volcanism. When life emerged, it consumed much of the free energy in the prebiotic atmosphere, replacing it with a smaller disequilibrium between biogenic waste gases: CO$_2$, CH$_4$, N$_2$ and liquid water. My finding that life destroyed much of the atmospheric free energy on prebiotic Earth contrasts the traditional view, proposed by \citet{Lovelock_1965}, that big disequilibrium should be generally associated with life. The early Earth suggests a more subtle relationship between life and atmospheric free energy.

I suggest that the disequilibrium-life relationship can be understood by considering the ``edibility'' of the disequilibrium in terms of reaction activation energy. Life consumed the prebiotic disequilibrium between H$_2$ and CO$_2$ or CO and H$_2$O because the reactions combining these species had relatively small activation energy barriers that could be overcome by enzymes. In contrast, the free energy present in Earth's atmosphere and ocean since the origin of life had big activation energy barriers that were insurmountable by biological catalysis. In other words, the prebiotic disequilibrium was ``edible'' and the disequilibrium present since biopoiesis was not. On this basis, I argue that big ``edible'' disequilibrium (e.g., the coexistence of volcanic H$_2$ and CO$_2$ or CO and H$_2$O) should be considered an anti-biosignature in exoplanet atmospheres. Life on an inhabited planet would consume this free lunch.

Life's presence or absence on an exoplanet cannot be definitively deduced by mere detections of ``edible'' or ``inedible'' chemical disequilibrium. We must also consider the surface fluxes of biogenic gases, and whether they might be mimicked by abiotic processes. This reality motivated Chapter \ref{ch:3}, which considers the surface fluxes of methane required to sustain the CH$_4$-CO$_2$ disequilibrium biosignature characteristic of the Archean Earth, and whether these fluxes might be imitated by magmatic volcanic outgassing on an exoplanet. Over a wide parameter space, my model of volcanic outgassing suggests that big volcanic methane fluxes comparable to biological fluxes are unlikely. In the rare circumstances where volcanic methane appears possible in our model, volcanoes also produce large amounts of carbon monoxide. Therefore, a genuine CH$_4$-CO$_2$ biosignature should most likely coincide with the lack of atmospheric CO.

Chapters \ref{ch:2} and \ref{ch:3} consider the biosignatures and anti-biosignatures of the Archean and late Hadean eons. Chapter \ref{ch:4} investigates atmospheric O$_2$, Earth's most renowned sign of life, during the Great Oxidation Event around 2.4 billion years ago. In this chapter, I use a novel time-dependent photochemical model to show that O$_2$ is unstable for concentrations between $10^{-8}$ to $10^{-4}$ mixing ratio. Natural perturbations to the atmosphere (e.g. volcanic eruptions, or changes in climate) drive this instability because they can cause O$_2$ between $10^{-8}$ and $10^{-4}$ mixing ratio to change by $\sim 4$ order of magnitude. This instability explains geologic evidence of O$_2$ oscillations preserved in the geologic record \citep{Poulton_2021}. Furthermore, the possibility of unstable O$_2$ requires that oxygen was bigger than $10^{-4}$ mixing ratio in the mid-Proterozoic eon. If O$_2$ concentrations were lower (e.g. $10^{-5}$ mixing ratio), then the natural perturbations to Earth would cause O$_2$ to occasionally drop below $10^{-7}$ mixing, which would be incompatible with the geologic record of sulfur isotopes. My new constraint on Proterozoic O$_2$ ($> 10^{-4}$ mixing ratio) has implications for the detectability of the Proterozoic biosphere if it were to manifest on an exoplanet.

Chapters \ref{ch:5} and \ref{ch:6} 
