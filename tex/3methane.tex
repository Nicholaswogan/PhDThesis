
\section*{\centering Summary}

The disequilibrium combination of abundant methane and carbon dioxide has been proposed as a promising exoplanet biosignature that is readily detectable with upcoming telescopes such as the James Webb Space Telescope. However, few studies have explored the possibility of non-biological CH$_4$ and CO$_2$ and related contextual clues. Here, we investigate whether magmatic volcanic outgassing on terrestrial planets can produce atmospheric CH$_4$ and CO$_2$ with a thermodynamic model. Our model suggests that volcanoes are unlikely to produce CH$_4$ fluxes comparable to biological fluxes. Improbable cases where volcanoes produce biological amounts of CH$_4$ also produce ample carbon monoxide. We show, using a photochemical model, that high abiotic CH$_4$ abundances produced by volcanoes would be accompanied by high CO abundances, which could be a detectable false positive diagnostic. Overall, when considering known mechanisms for generating abiotic CH$_4$ on terrestrial planets, we conclude that observations of atmospheric CH$_4$ with CO$_2$ are difficult to explain without the presence of biology when the CH$_4$ abundance implies a surface flux comparable to modern Earth's biological CH$_4$ flux. A small or negligible CO abundance strengthens the CH$_4$+CO$_2$ biosignature because life readily consumes atmospheric CO, while reducing volcanic gases likely cause CO to build up in a planet's atmosphere. Furthermore, the difficulty of volcanically-generated CH$_4$-rich atmospheres suitable for an origin of life may favor alternatives such as impact-induced reducing atmospheres.

\section{Introduction} \label{sec:intro}

