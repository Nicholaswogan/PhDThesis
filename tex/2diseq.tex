\section*{\centering Summary}

Chemical disequilibrium in exoplanetary atmospheres (detectable with remote spectroscopy) can indicate life. The modern Earth's atmosphere-ocean system has a much larger chemical disequilibrium than other solar system planets with atmospheres because of oxygenic photosynthesis. However, no analysis exists comparing disequilibrium on lifeless, prebiotic planets to disequilibrium on worlds with primitive chemotrophic biospheres that live off chemicals and not light. Here, we use a photochemical-microbial ecosystem model to calculate the atmosphere-ocean disequilibria of Earth with no life and with a chemotrophic biosphere. We show that the prebiotic Earth likely had a relatively large atmosphere-ocean disequilibrium due to the coexistence of water and volcanic H$_2$, CO$_2$, and CO. Subsequent chemotrophic life probably destroyed nearly all of the prebiotic disequilibrium through its metabolism, leaving a likely smaller disequilibrium between N$_2$, CO$_2$, CH$_4$, and liquid water. So, disequilibrium fell with the rise of chemotrophic life then later rose with atmospheric oxygenation due to oxygenic photosynthesis. We conclude that big prebiotic disequilibrium between H$_2$ and CO$_2$ or CO and water is an anti-biosignature because these easily metabolized species can be eaten due to redox reactions with low activation energy barriers. However, large chemical disequilibrium can also be a biosignature when the disequilibrium arises from a chemical mixture with biologically insurmountable activation energy barriers, and clearly identifiable biogenic gases. Earth's modern disequilibrium between O$_2$, N$_2$, and liquid water along with minor CH$_4$ is such a case. Thus, the interpretation of disequilibrium requires context. With context, disequilibrium can be used to infer dead or living worlds.

\section{Introduction}

It will soon be possible to look for biosignature gases in exoplanet atmospheres with telescopes. Within several years, the James Webb Space Telescope (JWST) will measure the composition of exoplanet atmospheres with transit spectroscopy \citep{Fischer_2016,Gaudi_2019}. Ground-based telescopes, such as the Extremely Large Telescope, will also play a role in the spectroscopic search for life by the mid 2020s