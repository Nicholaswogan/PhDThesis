\abstract{

The ancient Earth atmosphere is our only example for how a microbial biosphere impacts planetary atmospheres and is therefore a critical asset to the spectroscopic search for life on exoplanets. Additionally, for a subaerial origin of life, the nature of the earliest Earth atmosphere determines the environmental conditions under which life began. However, our understanding of the early Earth is shrouded by deep time; very few clues to its composition, climate and biosphere have been preserved over billions of years. To complement the sparse geologic record, this thesis uses thermodynamic, photochemical, and climate models to better understand the atmospheres of early Earth to inform the search for life on exoplanets and improve our understanding of the origin of life.

In Part \ref{pt:1} of this dissertation, I investigate atmospheric chemical disequilibrium anti-biosignatures, as well as methane and oxygen biosignatures during the Archean (4.0 - 2.5 Ga) and Proterozoic (2.5 - 0.54 Ga) eons. By modeling the change in Earth's atmospheric composition when life first began, I argue that the disequilibrium coexistence of atmospheric H$_2$ and CO$_2$ or CO and water vapor is an anti-biosignature if observed on an exoplanet because these easily metabolized species should be consumed if life was present. Next, I estimate the likelihood of volcanism on an exoplanet mimicking the CH$_4$+CO$_2$ biosignature characteristic of the Archean Earth. I find that significant volcanic methane is unlikely, but, if possible, could be identified by observations of atmospheric CO because volcanoes that produce CH$_4$ should also make CO. The final Chapter in Part \ref{pt:1} argues that atmospheric oxygen, Earth's most recognizable biosignature gas, was unstable during the Great Oxidation Event ($\sim 2.4$ Ga). I also better constraint O$_2$ levels during the Proterozoic eon, which has implications for the detectability of O$_2$ on an exoplanet if it was like the ancient Earth.

Part \ref{pt:2} explores how Earth's Hadean (4.5 - 4.0 Ga) atmosphere may have influenced the origin of life. Specifically, I use atmospheric models to estimate the HCN and HCCCN produced in the Hadean atmosphere in the wake of large asteroid impacts. Both HCN and HCCCN are critical ingredients in ``RNA world'' origin of life hypotheses. Simulations show that asteroid impacts make transient H$_2$- and CH$_4$-rich atmospheres that persist for millions of years, until hydrogen escapes to space. I find that impacts larger than between $5 \times 10^{20}$ to $4 \times 10^{21}$ kg (570 to 1330 km diameter) produce sufficient atmospheric CH$_4$ to cause ample HCN and HCCCN photochemical production and rainout to the surface, while smaller impacts produce negligible amounts of origin-of-life molecules. The second chapter of Part \ref{pt:2} places these results in the context of Earth's impact history. I estimate when $5 \times 10^{20}$ to $4 \times 10^{21}$ kg impacts most likely occurred on the early Earth to shed light on when life began if it did so in an impact-drive scenario. 

}