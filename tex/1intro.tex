
Past scientific exploration of the cosmos has delivered us repeated lessons in humility. Before Copernicus, humans perceived Earth to be at the center of the universe. Today, we know that rocky planets the size of Earth are common among the hundred-billion stars in our galaxy \citep{Burke_2015}. But science has not yet fully addressed Earth's apparent remaining cosmic uniqueness: Earth conceived and harbors life. Astrobiology attempts to complete our lessons in humility by trying to determine life's prevalence or rarity in the universe. 

Existing and future telescopes will allow for the search for life on planets orbiting other stars. The James Webb Space Telescope has begun to observe the atmospheres of small rocky planets and is sensitive enough to detect biogenic waste gases in some circumstances \citep{KrissansenTotton_2018_detect}. Additionally, the Decadal Survey on Astronomy and Astrophysics has recommended NASA construct the Habitable Worlds Observatory in the next few decades with a primary goal of characterizing exoplanet atmospheres and detecting biosignatures.

Even with pristine observations of exoplanet atmospheres, confirming the presence or absence of life will be challenging. Gases that are typically associated with biology, like O$_2$, could potentially build-up in exoplanet atmospheres without life from geologic or photochemical processes \citep{KrissansenTotton_2021_oxygen,Meadows_2018}. Additionally, given that exoplanets are light-years away we will not have the opportunity to visit them with spacecrafts to directly confirm or reject telescopic observations. Detecting life remotely will require a deep understanding of how life can influence an atmosphere, and the ways in which planetary evolution produces false-positive biosignatures.

To aid our understanding for how life might impact an exoplanet atmosphere, we can look to the co-evolution of Earth and its biosphere for the past 4 billion years. Earth's inhabited history has three main chapters with distinct atmospheres: The Archean (4.0 - 2.5 Ga), Proterozoic (2.5 - 0.54), and the modern Earth (0.54 - 0.0 Ga). Each era is an example atmosphere-biosphere interaction that might manifest on other worlds.

Furthermore, how life began on the early Earth informs the search for life elsewhere. If the environmental conditions that led to life's origin on the early Earth were unique and uncommon on young exoplanets, then perhaps future telescopes are unlikely to detect spectral signatures of life. However, if instead life's emergence is an inevitable side effect of early planetary evolution, then future campaigns searching for life are much more likely to be successful.

\section{Thesis outline}

Part \ref{pt:1} of this thesis studies the interaction between the atmosphere and biosphere on the early Earth to inform the search for life on exoplanets. First, in Chapter \ref{ch:2}, I investigate atmospheric chemical disequilibrium as a biosignature and anti-biosignature. The relationship between disequilibrium and life was first explored by \citet{Lovelock_1965} for the purposes of detecting microbes on the solar system planets. Since then, \citet{Sagan_1993} observed that the disequilibrium coexistence of CH$_4$ and O$_2$ in Earth's modern atmosphere is a sign of life. CH$_4$ and O$_2$ rapidly annihilate each other through photochemical reactions, but the disequilibrium persists because of biological replenishment. More recently, \citet{KrissansenTotton_2016} rigorously quantified disequilibrium as the Gibbs free energy in the atmosphere-ocean system. They used multiphase thermodynamic calculations to show that modern Earth has an anomalously big biologically produced disequilibrium compared to other solar system planets.

Chapter \ref{ch:2} argues that disequilibrium is sometimes an anti-biosignature instead of a sign of life. Life is powered by chemical free energy. Therefore, in some circumstances, life should reduce the disequilibrium in its environment through its metabolism. I investigate this possibility by estimating how chemical disequilibrium changed when life first began on the early Earth. I demonstrate that Earth's atmosphere-ocean disequilibrium lowered when microbial life first appeared because such life likely consumed a pre-existing free energy in the atmosphere caused by volcanic outgassing. This chapter concludes by clarifying when atmospheric disequilibrium is a sign of life, and when it is instead an anti-biosignature.

In combination with previous research \citep{KrissansenTotton_2018_diseq}, Chapter \ref{ch:2} also established that the biogenic disequilibrium between atmospheric CH$_4$ and CO$_2$ was likely present through the entire Archean eon (4.0 - 2.4 Ga). Carbon dioxide was generated by volcanism and regulated by the geologic carbon cycle, while CH$_4$ concentrations resulted from methanogenic microbial life \citep{Catling_2020}. Thus, the disequilibrium between methane and carbon dioxide might be a compelling biosignature if identified spectroscopically in an exoplanet atmosphere \citep{KrissansenTotton_2018_diseq}. However, few studies have explored the possibility of non-biological CH$_4$ and CO$_2$ which might result in a false-positive detection of life.

Chapter \ref{ch:3} uses a model of volcanic outgassing to determine whether volcanoes on a terrestrial planet can produce atmospheric CH$_4$ and CO$_2$, thus mimicking Earth's Archean biosignature. In general, I find that significant volcanic methane is unlikely, but, if possible, could be identified by observations of atmospheric CO. Atmospheric CO is a false-positive diagnostic because volcanoes that produce CH$_4$ should also produce CO. Overall, when considering known mechanisms for generating abiotic CH$_4$ on rocky planets, I conclude that observations of atmospheric CH$_4$ with CO$_2$ are difficult to explain without the presence of biology when the CH$_4$ abundance implies a surface flux into the atmosphere comparable to modern Earth's biological CH$_4$ flux.

Chapters \ref{ch:2} and \ref{ch:3} investigate life's influence on the atmosphere in the late Hadean and early Archean atmosphere. However, arguably life's most profound impact on the atmosphere occurred much later, around 2.4 billion years ago, when oxygenic photosynthetic life (combined with several other factors) caused the Great Oxidation Event (GOE). The GOE marks the first appearance of the modern Earth's most identifiable biogenic gas: atmospheric oxygen.

In Chapter \ref{ch:4} I use a novel time-dependent one-dimensional photochemical model to simulate the Great Oxidation Event on Earth and find that the transition from an anoxic to an oxic atmosphere takes only $10^2$ to $10^5$ years. My model also suggests that O$_2$ between $\sim 10^{-8}$ and $\sim 10^{-4}$ mixing ratio is unstable, and prone to rapid change over geologic time. I suggest that this instability can explain geologic evidence for fluctuating O$_2$ levels during the Great Oxidation Event \citep{Poulton_2021}. This result is significant because it challenges an existing paradigm \citep{Goldblatt_2006} that the first rise of oxygen was a stable and irreversible event. I also argue that instability requires that the mid-Proterozoic (1.8 - 0.8 Ga) O$_2$ levels were bigger than $10^{-4}$ mixing ratio, because smaller O$_2$ levels would be unstable and ultimately incompatible with the geologic record of sulfur isotopes. Therefore, these results have implications for the detectability of Earth's O$_2$ biosignature during the Proterozoic.

Part \ref{pt:2} of this thesis addresses the broader question of what conditions lead to an origin of life on a planet. Understanding life's beginning on Earth, and whether it is likely to occur on an exoplanet is a pre-requisite to finding life elsewhere. A remote detection of a biosignature would be much less reputable if the planet did not experience the conditions that are thought to be important for biopoiesis on the early Earth \citep{Catling_2018}.

How life began on Earth is currently unknown, but one leading hypothesis suggests that strands of RNA were some of the first self-replicating molecules which perhaps ultimately evolved to become early life. In this scenario, RNA needs to be produced without life on the early Earth. Chemists have argued for several possible schemes, but all pathways require nitriles like HCN and HCCCN. Geochemical evidence suggests that volcanoes produced a CO$_2$- and N$_2$-rich Hadean atmosphere that would not generate essential prebiotic nitriles \citep{Holland_1984}. Iron-rich asteroid impacts could have solved this problem because they may have transiently reduced the entire atmosphere, allowing HCN and HCCCN to form photochemically \citep{Zahnle_2020}. Chapter \ref{ch:5} investigates this possibility by using novel coupled photochemical-climate models to simulate the Hadean atmosphere after massive impacts thereby quantify prebiotic nitrile production. Overall, I find that atmospheres after impacts are H$_2$-, CH$_4$- and NH$_3$-rich, and HCN and HCCCN are produced photochemically for impacts larger than between $5 \times 10^{20}$ to $4 \times 10^{21}$ kg (570 to 1330 km diameter). Smaller impacts chemically alter the Hadean atmosphere, but not by an amount that allows for the photochemical formation of both HCN and HCCCN. 

The final chapter in this thesis (Chapter \ref{ch:6}) combines the results from Chapter \ref{ch:5}, with Monte Carlo simulations of Earth's impact history to estimate the likelihood and timing of life's emergence in an impact-driven scenario. I use the Lunar cratering record, and the abundance of highly siderophile elements in Earth's mantle to simulate the many possible impact histories that could have occurred on the early Earth. A fraction of the modeled impact histories experience impactors of sufficient mass to produce significant origin of life molecules, perhaps causing the emergence of life. However, in some cases life would not persist because the post-impact reducing atmosphere is subsequently followed by an ocean-vaporizing impact that would sterilize the planet. By considering the fraction of stochastic impact realizations that successfully initiate or fail to initiate life, I estimate the probability of life beginning if Earth history was rerun. Additionally, for impact histories where life is not destroyed by late ocean vaporization, I compute a probability distribution for when life began in an impact-driven scenario by considering the timing of the last post-impact reducing atmosphere.
